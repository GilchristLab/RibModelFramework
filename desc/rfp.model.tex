\documentclass{article}

\usepackage{fullpage}
\usepackage{datetime} %provides \currenttime command
\usepackage{xspace}
\usepackage{amssymb,amsfonts,amsmath}
\usepackage{natbib}
\usepackage[doublespacing]{setspace}
\usepackage{tabulary} %Allows multirowed columns whose width is calculated automatically. Similar to tabularx but allows column width to vary between column types and calculated based on content of columns.  See p 253 of Mittelbach and Gossens LaTeX Companion, 2nd Ed


\newcommand{\waitTerm}{\ensuremath{w}\xspace}
\newcommand{\wc}{\ensuremath{\waitTerm^c}\xspace}
\newcommand{\wgi}{\ensuremath{\waitTerm_{g,i}}\xspace}
\newcommand{\wgj}{\ensuremath{\waitTerm_{g,j}}\xspace}
\newcommand{\wgc}{\ensuremath{\wc_{g}}\xspace}
\newcommand{\alphai}{\ensuremath{{\alpha_i}}\xspace}
\newcommand{\alphaj}{\ensuremath{{\alpha_j}}\xspace}
\newcommand{\alphac}{\ensuremath{{\alpha_c}}\xspace}
\newcommand{\alphacprime}{\ensuremath{{\alpha_c^\prime}}\xspace}
\newcommand{\alphacvec}{\ensuremath{{\vec{\alpha}_c}}\xspace}
\newcommand{\lambdac}{\ensuremath{{\lambda_c}}\xspace}
\newcommand{\lambdacprime}{\ensuremath{{\lambda_c^\prime}}\xspace}
\newcommand{\lambdacprimevec}{\ensuremath{{\vec{\lambda}_c^\prime}}\xspace}
\newcommand{\lambdai}{\ensuremath{{\lambda_i}}\xspace}
\newcommand{\lambdaiprime}{\ensuremath{{\lambda_i^\prime}}\xspace}
\newcommand{\lambdaiprimevec}{\ensuremath{{\vec{\lambda}_i^\prime}}\xspace}
\newcommand{\lambdaj}{\ensuremath{{\lambda_j}}\xspace}
\newcommand{\lambdajprime}{\ensuremath{{\lambda_j^\prime}}\xspace}
\newcommand{\lambdajprimevec}{\ensuremath{{\vec{\lambda}_j^\prime}}\xspace}



%Nonsense error terms
\newcommand{\nseTerm}{\ensuremath{v}\xspace}
\newcommand{\vc}{\ensuremath{\nseTerm^c}\xspace}
\newcommand{\vgi}{\ensuremath{\nseTerm_{g,i}}\xspace}
\newcommand{\vgj}{\ensuremath{\nseTerm_{g,j}}\xspace}
\newcommand{\vgc}{\ensuremath{\vc_{g}}\xspace}
\newcommand{\betac}{\ensuremath{{\beta_c}}\xspace}
\newcommand{\muc}{\ensuremath{{\mu_c}}\xspace}


\newcommand{\pgi}{\ensuremath{{p_{g,i}}}\xspace}
\newcommand{\Pgi}{\ensuremath{{P_{g,i}}}\xspace}
\newcommand{\Pgj}{\ensuremath{{P_{g,j}}}\xspace}
\newcommand{\Pgc}{\ensuremath{{P_{g}^c}}\xspace}
\newcommand{\ngc}{\ensuremath{{n_{g}^c}}\xspace}
\newcommand{\ns}{\ensuremath{{n_s}}\xspace}
\newcommand{\mg}{\ensuremath{{m_g}}\xspace}
\newcommand{\Mg}{\ensuremath{{M_g}}\xspace}
\newcommand{\phig}{\ensuremath{{\phi_g}}\xspace}
\newcommand{\phige}{\ensuremath{{\phi_g^e}}\xspace}
%\newcommand{\lambdagi}{\ensuremath{{\lambda_{g,i}}}\xspace}
%\newcommand{\lambdagc}{\ensuremath{{\lambda_{g}^c}}\xspace}
\newcommand{\kappag}{\ensuremath{{\kappa_{g}}}\xspace}
\newcommand{\Ztheta}{\ensuremath{{Z}}\xspace}
\newcommand{\mRNAg}{mRNA$_g$\xspace}
\newcommand{\Yg}{\ensuremath{{Y_{g}}}\xspace}
\newcommand{\Ygi}{\ensuremath{{Y_{g,i}}}\xspace}
\newcommand{\Ygc}{\ensuremath{{Y_{g}^c}}\xspace}
\newcommand{\Kg}{\ensuremath{{K_{g}}}\xspace}
%\newcommand{\qcg}{\ensuremath{{q_{c,g}}}\xspace}
\newcommand{\Lik}{\ensuremath{\text{L}}\xspace}
\newcommand{\setG}{\ensuremath{\mathbb{G}}\xspace}
\newcommand{\setC}{\ensuremath{\mathbb{C}}\xspace}
%\newcommand{\}{\ensuremath{\}\xspace}
%\newcommand{\}{\ensuremath{\}\xspace}

\newcommand{\mgvec}{\ensuremath{{\Vec{\mg}}}\xspace}
\newcommand{\Ygcvec}{\ensuremath{{\Vec{\Ygc}}}\xspace}
\newcommand{\kappagvec}{\ensuremath{{\Vec{\kappag}}}\xspace}
\newcommand{\ngcvec}{\ensuremath{{\Vec{\ngc}}}\xspace}
\newcommand{\phigvec}{\ensuremath{{\Vec{\phig}}}\xspace}
\begin{document}


\subsection*{History}
\begin{itemize}
\item Initial model  by mikeg on 6/10/15.
\item NSE model added by mikeg on 7/21/15.
\item Compiled on \today at \ \currenttime
 \end{itemize}
\subsection*{Goal}
\label{goal}
Create sensible model for interpreting RPF data with a special interest in working with data used in \citet{PopEtAl14}.

\section*{Pausing Time Model Definition}
\subsection*{Calculating the Likelihood of a sample}
We are interested in calculating the probability of observing a ribosome footprint (RFP) experimentally.
We assume there is a pool of RFP generated from the transcriptome, that the mRNAs in this pool are at close to steady state in terms of ribosome initiation and completion of translating a transcript.

Beginning by considering a single mRNA molecule transcribed from gene $g$, the probability a ribosome is at position $i$ of this mRNA molecule, \pgi,  is simply,
\begin{equation} \label{eq:defpgi}
\pgi  \propto \kappag \wgi
\end{equation}
where \kappag is a rate constant scales the average rate at which ribsomes intercept and initiate translation of an mRNA molecule from gene $g$, \mRNAg, under the experimental conditions used.
Formally, the initiation rate is determined by $\kappag \times r$, where  $r$ is the density of ribosomes in the cell.
However, we assume all mRNAs are equally accessible to ribosomes so, as a result, $r$ will cancel out in the following equation and, as a result, we ignore it throughout.
Additionally, \wgi is the average waiting, pausing, or dwell time of a ribosome at position $i$ of mRNA from gene $g$.  
Derivation of equation \ref{eq:defpgi} is straight forward, but can also be found \citet{GilchristAndWagner06} equation (20) with the nonsense error rate = 0 and the ribosome recycling probability $<1$.
We can link it to \citet{PopEtAl14} work by noting the ribsome flux $J_g$ on an individual \mRNAg is $J_g = r \kappag$.

Biologically speaking, \wgi values for the same codon are not independent.
The values of \wgi for the relevant codon likely vary within a gene as a function of mRNA structure and other factors.
To capture this variation, we will assume that for when the codon at position $i$ is of type $c$, $\wgi \sim \text{Gamma}(\alphac, \lambdac)$, where \alphac is the shape parameter, \lambdac is the scale parameter, and $E[\wgi] = \alphac \lambdac$.
Gene specific effects could also be incorporated since  mRNA structures which interfere with efficient translation likely declines with expression level.
As a result of this assumption, we can use a Negative-Binomial (NB) distribution model to analyze the RFP data.


Assuming independence in sampling, the total probability of a randomly selected footprint is from position $i$, $\Pgi$, is
\begin{equation} \label{eq:defPgi}
\Pgi = \pgi \mg/\Ztheta 
\end{equation}
where \mg is the density of \mRNAg in the cell and \Ztheta is a partition function that ensures our sampling probabilities across the transcriptome sums to 1 and is defined as,
\begin{equation}
  \label{eq:defZtheta}
  \Ztheta = \sum_g \mg \left(\sum_i \pgi\right).
\end{equation}
Equations (\ref{eq:defPgi}) and (\ref{eq:defZtheta}) indicate that our choices of time and volume units for \kappag and \mg are irrelevant and we can only estimate their values relative to one another.

%We will do our best to avoid calculting \Ztheta directly and, instead, treat it as an unknown variable and attempt to sample from its posterior distribution.

Letting $\ns$ be the footprint sample size, i.e.~the number of footprints examined, where $\ns \gg 1$ and $\Pgi \ll 1$, then we can approximate the probability of observing $\Ygi$ samples of a footprint in \mRNAg at position $i$ using a Poisson distribution with a sampling rate of $\ns \Pgi$.
Note that deep sequencing may violate the sampling with replacement assumption of the Poisson distribution.
Given our assumption about the distribution of $\pgi$,  $\Ygi \sim \text{NB}(x = \alphac, p = \kappag \mg /(\lambdacprime + \kappag \mg)$ where $\lambdacprime = \lambdac \Ztheta/\ns$.
Explicitly,
\begin{equation}
  \label{eq:distYgSite}
  \Pr\left(\Ygi \middle| \alphac, \lambdacprime, \mg, \kappag\right) = \frac{\Gamma\left(\alphac + \Ygi\right)}{\Gamma\left(\alphac\right) \Ygi!} 
  \left(\frac{\mg \kappag}{\lambdacprime + \mg \kappag}\right)^\Ygi \left(\frac{\lambdacprime}{\lambdacprime + \mg \kappag}\right)^\alphac
\end{equation}
Note that $\mg$ and \kappag are gene $g$ and environment specific terms which can be equated to the equilibrium protein synthesis rate $\phi_g$ for gene $g$ under the experimental conditions, i.e.~$\phig = \kappag \mg$.
The composite parameter \lambdacprime consists of the codon specific scale term \lambdac and  the ratio of \Ztheta to \ns, two genome wide parameters.
%The term \ns/\Ztheta can be interpreted as the sampling efficiency, i.e.~what proportion of the RFP state space is sampled.
%If $\ns/\Ztheta < 1$ then sampling is sparse.

Given the properties of the NB and the fact that most codons appear within a gene's ORF multiple times, the likelihood of the parameters given the total number of RFP observed derived from codons of type $c$ in gene $g$, $\Ygc = \sum_{i=\in c} \Ygi$, is,
\begin{align}
  \Lik\left(\phig, \alphac, \lambdacprime \middle| \Ygc, \ngc\right) &= \frac{\Gamma\left(\ngc \alphac + \Ygc\right)}{\Gamma\left(\ngc \alphac\right) \Ygc!} 
  \left(\frac{\phig}{\lambdacprime + \phig}\right)^\Ygc \left(\frac{\lambdacprime}{\lambdacprime + \phig}\right)^{\ngc \alphac}\\
\label{eq:distYgCodon}&= \frac{\Gamma\left(\alphacprime + \Ygc\right)}{\Gamma\left(\alphacprime\right) \Ygc!} 
  \left(\frac{\phig}{\lambdacprime + \phig}\right)^\Ygc \left(\frac{\lambdacprime}{\lambdacprime + \phig}\right)^{\alphacprime}
\end{align}
where $\ngc$ is the number of times codon $c$ is found in the ORF of gene $g$ and $\alphacprime = \ngc \times \alphac$.
The total Likelihood of the data is
\begin{equation}
  \Lik\left(\phigvec, \alphacvec, \lambdacprimevec \middle| \Ygcvec, \ngcvec\right) = \prod_{g \in \setG} \prod_{c \in \setC}  \Lik\left(\phig, \alphac, \lambdacprime \middle| \Ygc, \ngc\right)
\end{equation}
Note that using RFP data alone, $\kappag \times \mg = \phi$ and $\lambdac \Ztheta$ are only identifiable as joint parameters. 
(Although it seems like you should be able to calculate \Ztheta post-hoc from the state of the chain.)
Most standard libraries require that the $x$ parameter in a NB be discrete, which in this case it is not.
Thus to simulate \Yg values based on equations (\ref{eq:distYgSite}) or (\ref{eq:distYgCodon}), first pull $L$ from Gamma(\alphac, \lambdacprime) or Gamma(\alphacprime, \lambdacprime), then pull $Y$ from Poisson($L$).
  
\citet{PopEtAl14} provide RNA-Seq based counts of mRNA abundances, \Mg, in addition to RFP counts.
If we assume that $\Mg \sim \text{Poisson}(\ns \mg)$, we can easily combine both the RFP and the mRNA counts together and estimate $\kappag$ and $\mg$ separately.
Alternatively, we could estimate the composite \kappag \mg parameter using the RPF data and then analyze the those results using the \Mg data.

\subsection*{Simulation}
Although our Likelihood function can be described using a Negative Binomial distribution, because $x = \alpha$ and $\alpha$ is not discrete, we cannot use standard NB routines to simulate our data.
Instead we need simulate in two steps.
First, we generate $W_c$  from $\text{Gamma}(\alphacprime, \lambdacprime)$ and then we generate $\Ygc$ from $\text{Poisson}\left(\phi_g W_c\right)$


\section*{Pausing Time with Nonsense Error Model Definition}
The flux equation, Equation (\ref{eq:defpgi}), no longer holds when nonsense errors (NSE) are possible.
Instead, following \citet{GilchristAndWagner06}, we have the conditional probability,
\begin{equation}
\label{eq:defpgiNse}
\pgi|\wgi \propto \kappag \sigma(i-1) \frac{\wgi \vgi}{\wgi + \vgi}
\end{equation} 
Where, as before, \kappag is the translation initiation rate constant per \mRNAg, \wgi is the waiting time to elongate codon $i$, and  \wgi is 1 over the codon elongation rate ($1/c_i$ in rate based, rather than waiting time, terminology) and \vgi is the NSE 'wait' time, i.e.~1 over the NSE rate ($1/b_i$ in rate based terminology), and $\sigma(i-1)$ is the probability a ribosome that initiates translation will reach the $i$th codon.
Note that because the waiting time to a NSE, \vgi, is so much greater than the elongation waiting time \wgi we can ignore the actual variation in \vgi between codons of the same type (this is easier to understand if you consider $0 < b_i \ll 1$).
Thus while we allow \vgi to be codon specific, we treat each of these values as fixed.
Further, again because $\vgi \gg \wgi$, we can approximate $\frac{\wgi \vgi}{\wgi + \vgi}$ as $\wgi$ based on a Taylor series expansion around $1/\vgi = 0 $.
Thus,
\begin{equation}
\label{eq:defpgiNseApprox}
\pgi|\wgi \propto \kappag \sigma(i-1) \wgi
\end{equation} 
which is equivalent to the simple pausing time calculation except $\pgi$ is reduced by $\sigma(i-1)$.


The function $\sigma(i-1)$ depends on the probability of successful elongation at the $i-1$ upstream codons.
Letting $f(\alpha, \lambda)$ represent the PDF of the Gamma distribution, the codon specific elongation probability is
\begin{align}
\label{eq:defElongPr}
\Pr(\text{Elongation at position $j$}) &= \int_0^\infty \frac{\vgj}{\wgj + \vgj} f\left(\wgj | \alphaj, \lambdaj\right) d\wgj\\
 &= \exp\left[\lambdaj \vgj\right] E_{n = \alphaj}\left(\lambdaj \vgj\right)
\end{align}
where $E_n(x)$ is the exponential integral function.
Assuming independence in \wgi between positions, 
  \label{eq:defSigma}
\begin{align}
  \sigma(i) &=  \exp\left[\sum_{j=1}^i \lambdaj \vgj\right]  \prod_{j=1}^{i} E_{\alphaj}\left(\lambdaj \vgj\right)
\end{align}
Note that these \lambdaj terms are equivalent as \lambdac, but not \lambdacprime.

Noting that the reasoning which led to equations \ref{eq:defPgi} and \ref{eq:defZtheta} for the pausing time model should still apply if we use Equation (\ref{eq:defpgiNse}) for \pgi, rather than Equation (\ref{eq:defpgi}). 
As a result,
\begin{equation}
  \label{eq:distYgSiteNse}
  \Pr\left(\Ygi \middle| \alphai, \lambdaiprime, \mg, \kappag, \sigma(i-i)\right) = \frac{\Gamma\left(\alphai + \Ygi\right)}{\Gamma\left(\alphai\right) \Ygi!} 
  \left(\frac{\mg \kappag \sigma(i-1)}{\lambdaiprime + \mg \kappag \sigma(i-1)}\right)^\Ygi \left(\frac{\lambdaiprime}{\lambdaiprime + \mg \kappag \sigma(i-1)}\right)^\alphai
\end{equation}
While the NSE model requires the likelihood for each position be calculated separately, the underlying terms for $\sigma(i-1)$ need only be calculated once per parameter evaluation since it is only how these terms are combined that varies between codons at different positions.

\subsection*{Simulation}
Unlike with the Pausing Time Model, we cannot aggregate codon specific RPF counts within a gene.
Instead we have to simulate each position and as before we simulate in two steps.
First, we generate $W_i$  from $\text{Gamma}(\alphai, \lambdaiprime)$ and then we generate $\Ygi$ from $\text{Poisson}\left(\phi_g \sigma(i-1) W_i\right)$


\section*{Parameter Definitions}
\label{paramDefs}
\setlength\tymin{30pt}  %minimum column width
%\setlength\tymax{200pt}  %maximum column width
\begin{table}[H!]
%  \begin{tabulary}{\textwidth}{|>{$}C<{$}|L|>{$}C<{$}|} %>{$}C<{$}| should type set the column in math mode. Use \text{} to switch to text mode.
  \begin{tabulary}{\textwidth}{|C|L|C|} 
       \hline
       & Definition & {Units}\\ \hline \hline 
    \wgi & Ribosome waiting/pausing/dwell time at codon position $i$ in gene $g$ & 1/t\\
    \alphac, \lambdac& Shape and rate parameter for distribution of waiting times for codon $c$ . The rate parameter is inversely related to average wait time, i.e.~for codon $c$ $E[\wgi] = \alphac/\lambdac$  & - \\
    \mg & Density of mRNA transcripts for gene $g$ in cytosol.& 1/{Vol}\\  
    \Mg & Observed mRNA counts from RNA-Seq data.             & 1/{Vol}\\  
    \kappag & Rate constant determining ribosome initiation rate per mRNA.  Function of diffusion of ribosomes, mRNA, and other factors. & $\frac{1}{\text{rib. mRNA Vol} t}$\\
    \phig & Steady state protein production rate. Equal to $\mg \times \kappag$. &\\ 
    \pgi & Probability a ribosome is found at position $i$ on an mRNA transcript from gene $g$ when translation initiation and completion of mRNA is at steady state. &\\%$\pgi \prop \kappag \wgi$ & \\
    \Pgi & Probability of observing a footprint for position $i$ of mRNA from gene $g$.& \\%$\Pgi = \left(\pgi \mg\right)/\Ztheta$.& \\
    \Pgc & Probability of observing a footprint for codon $c$ of mRNA from gene $g$. &\\%Generally, $\Pgc = \sum_{i \in c} \Pgi$. Assuming constant elongation rates for a codon type within a transcript, i.e.$\wgi = \wgj = \wgc | \{i,j\} \in c$ gives, $\Pgc = \Pgi \ngc$.& \\
    \Ztheta& Partition function which scales the codon footprint sampling probabilities \Pgi summed across $i$ and $g$ equals 1. &\\
    \ngc & Number of codons type $c$ in mRNA of gene $g$. &  \\
    \ns & RPF sample size. That is, number of fragments in dataset.& \\
    \Ygi & Number of RFP observed for position $i$ in gene $g$ &\\
    \Ygc & Number of RFP observed for codon $c$ in gene $g$ &\\
    \lambdacprime& Composit parameter equal to $\lambdac Z/\ns$ &\\
    $\pi_j$  & Prior probability for parameter $j$. & \\ \hline
  \end{tabulary}
   \caption{Table of model parameters}
  \label{tab:modelParam}
\end{table}


\bibliographystyle{natbib}
\bibliography{/home/mikeg/BiBTeX/bibliography.full}

\end{document}


%    \wgc & Average ribosome waiting time at codon of type $c$ in gene $g$.  
%    %Note if we assume that elongation rate is the same across codons of type $c$ within a gene, then $\wgc = \wgi \forall i \in c$.  
%    & 1/t \\
%    \wc & Average ribosome waiting time of codon $c$ across all genes.  Assume $\wgi \sim\text{Gamma}(\alphac, \lambdac)$ so that counts will follow Negative Binomial distribution. & 1/t\\
%    & Shape parameter for distribution of waiting times for codon $c$ where $E[\wc] = \alphac/ \lambdac$ & -  \\
%    r & Density of ribosomes within a cell's cytosol.  Assuming well mixed system and diffusion limited interactions between ribosomes and mRNAs & \small $\frac{\text{rib.}}{\text{Vol}}$\\ 
    %That is,  $\Ztheta = \sum_g \kappag \mg \left(\sum_i \pgi\right) $. & ? \\
    %\lambdagi & Poisson rate for observing footprint for codon at position $i$ of gene $g$ where $\lambdagi = \ns \Pgi$.& \\  
    %\lambdagc & Poisson rate for observing footprint for codon of type $c$ of gene $g$. $\lambdagc = \ns \sum_{i\in c} \Pgi = \ns \ngc \lambdagc$ if $\wgi = \wgj = \wgc$.  & \\    %If $\wgc = \wgi = \wgj | \{i,j\} \in c$ and $\wgc \sim \text{Gamma}(\alphac, \lambdac)$, then & \\
